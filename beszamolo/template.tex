%                                             -*- coding: utf-8 -*-
% Mindenkinek csak javasolni tudjuk, hogy latex-et használjon.
% Szakdolgozatnál vagy diplománál már egyértelműen kijönnek az
% előnyei a Worddel szemben.  Ennek ellenére ez a sablon messze nem
% tökéletes.  Ha valamit javítanál benne, kérlek, küld vissza, hogy
% hallgatótársaid is profitáljanak belőle.  Köszönöm.

% További nehézséget okoz, hogy a népszerű latex disztribúciók nem
% tartalmazzák a legújabb változatát a magyar.ldf-nek.  A szükséges
% fájlokat a sablon mellé bemásoltuk, de le is tölthetőek innen:
% http://www.math.bme.hu/latex/
%
%
%
\documentclass[a4paper,oneside]{article}
\usepackage[margin=3cm]{geometry}
% =================================================================
% Magyar nyelvi támogatás
%------------------------
% ###################
% Nyelvváltó parancsok:
%\selectlanguage{english}
%\selectlanguage{magyar}
% rövid angol beszúrás:  \foreignlanguage{english}{some english text}
% határozott névelők generálása ``magyar'' babel-el:
% argumentum+megfelelő határozott nevelő: \az{},\Az{}
% csak a megfelelő határozott nevelő: \az*{}, \Az*{}
% címkék: \aref{}, \aref*{}, képletekhez \aref()
%        \Aref{}, \Aref*{}, képletekhez \Aref()
% oldalak: \apageref{}, \apageref*{}
%        \Apageref{}, \Apageref*{}
% idézetek: \acite, \acite*, \Acite, \Acite*
% ###################
\usepackage[english,magyar]{babel} %vegyes nyelvi támogatás a
% magyar helyesírás ellenőrzéshez (ispell) és elválasztáshoz
\selectlanguage{magyar}

%=================================================================
% direkt ékezetes karakter beírás támogatás
%-------------------------------------------
\usepackage[T1]{fontenc}
\usepackage[utf8]{inputenc}
\usepackage{multirow} 
%================================================================
% Undorító dolog bitmappelt (Type III) betűtípust nézni a PDF-ben
% képernyőn. Az alapértelmezett Computer Modern font LaTex-ben
% bitmappelt, ezért használjunk Times fontot:
\usepackage{times}

%================================================================
% ha ábrát akarunk beemelni, akkor használjuk a graphicx/graphics
% csomagot és az \includegraphics[width=<width>]{abra.pdf} parancsot
\usepackage{graphicx} %for graphics
%kepek helye a gyokerhez(ehhez a file-hoz kepest) kepest
\graphicspath{{./figs/}}

%================================================================
% Kötelezően használjuk a hyperref csomagot, mert ezzel többek között 
%  kultúrált hyperlinkelt PDF-et lehet csinálni az alábbi
%  variációkban, különféle hyperref backend-ekkel:
%  pdflatex,dvipdfm,ps2pdf
% tapsztalataim szerint a MikTeX (Win32) a 'dvipdfm' konverzióval
% optimális  míg a teTeX (Linux/Solaris) jobb szereti a 'dvips' módszert
%------------------------------------
% pontosan egyet kommentezzünk be!!!!!!!
% értelemszerűen backend függően generáljunk dvi-ból PDF-et!!!
%------------------------------------
% A hyperref csomag az utolsó beolvasott csomag legyen, kivéve néhány
% problémás csomagot, pl. algorithm
%-----------
% ########################### FONTOS ###########################
% A hyperref hibásan működik a babel csomag 'magyar.ldf' fájljának
% 1.5-ös verziójánál korábbi változatával. 2004. februárjában a MikTeX
% és teTex disztribúciók még csak a v.1.4 verziót tartalmazták! A fájl
% aktuális verziója a BME Matematikai intézet LaTeX honlapjáról
% elérhető: http://www.math.bme.hu/latex/ 
% A lusták kedvéért a jelen sablon mellé is mellékelem:
% magyarlatex_0.01-2.tar.gz 
% ########################### FONTOS ###########################
%-----------
\usepackage[colorlinks=true]{hyperref}
\usepackage{amsmath}
\usepackage{mathrsfs}
\usepackage{tikz}
\usepackage{wrapfig}

\renewcommand{\theequation}{\arabic{section}.\arabic{equation}}

\newcommand{\inedge}[1]{E^-(#1)}
\newcommand{\outedge}[1]{E^+(#1)}

\counterwithin{equation}{section}

%%%%%%%%%%%%%%%%%%%%%%%%%%%%%%%%%%%%%%%%%%%%%%%%%%%%%%%%%%%%%%%%%%%
% Itt kezdődik maga a dokumentum
%%%%%%%%%%%%%%%%%%%%%%%%%%%%%%%%%%%%%%%%%%%%%%%%%%%%%%%%%%%%%%%%%%
\begin{document}
\input{onlabmacros} % Ez kell!!!
\markright{Apostagi Bálint (WAOIBU)} % egyoldalas fejléc!!!
%--------------------------------------------------------------------
% fedlap
%--------------------------------------------------------------------
\begin{titlepage}
%bme logo 
 \begin{figure}[h]
    \centering
      \includegraphics[width=12cm]{bme_logo}
  \label{fig:bme_logo}
  \end{figure}
  \thispagestyle{empty}
  %cím generálás
  \onlabcim

% \begin{center}
%   \begin{tabular}{ p{3cm} p{5cm} }
%   
%   Készítette: & Beszámoló Péter  \\
%   Neptun-kód: & BPOX43  \\
%   Ágazat: & Médiainformatika  \\
%   E-mail cím: & b.peter@onlab.hu  \\
%   Konzulens: & Dr. Péhádes István  \\
%   E-mail cím: & pehades@tmit.bme.hu  \\
%   Konzulens: & Doktor Andusz  \\
%   E-mail cím: & doktora@tmit.bme.hu  \\
%   
%   \end{tabular}
% \end{center}

 
  %\szerzo argumentumok: #1=Név, #2=Neptunkód, #3=szakirány, #4=email,#5 konzulens-1, #6 konzulens-1-email, #7 konzulens-2, #8 konzulens-2-email
  \onlabszerzo{Apostagi Bálint}{WAOIBU}{Mérnökinformatikus BSc}{apostagi.balint@edu.bme.hu}{Dr. Cinkler Tibor}{cinkler@tmit.bme.hu}{-}{}
 
 
%\feladatcim argumentuma a feladat rövid, 1 soros címe
  \feladatcim{Hálózati folyam modellezése és optimalizálása lineáris módszerekkel.} 

  %\feladatmaga argumentuma a feladat 1-2 bekezdésnyi ismertetése
  \feladatmaga{
    Legrövidebb út probléma megoldása
    IBM ILOG CPLEX Optimizer\footnote
    {
      Az IBM CPLEX a simplex módszerre épülő szoftver, lineáris kényszerekkel megadott optimalizálási feladatok megoldására. A szoftver elérhető az IBM honlapján:
       \href{https://www.ibm.com/products/ilog-cplex-optimization-studio/cplex-optimizer}
            {https://www.ibm.com/products/ilog-cplex-optimization-studio/cplex-optimizer}
    }
    használatával. Meghibásodás esetére optimális, a fő útvonallal éldiszjunkt védelmi útvonal számítása.
    A megoldások alkalmazása általános 5G hálózatra, amelyben mobil ezközök, gNodeB bázisállomások és egy
    átjáró\footnote
    {
      A hálózatot más hálózatokkal összekötő hálózati eszköz, általában az angol gateway kifejezés magyarítása.
    }
    szerepel.
    A megoldás fejlesztése, hogy az ki tudja használni az 5G hálózatokon támogatott többszörös csatlakoztathatóságot.
  }

 
  %\tanevfelev argumentumok:
  % #1=Tanév (xxxx/xx alakban), #2=félév (pont nélkül!)
  
  \tanevfelev{2023/24}{I}
 
\end{titlepage}

%==================================================================
\section{A laboratóriumi munka környezetének ismertetése,
     a munka előzményei és kiindulási állapota}
\label{sec:kornyezet}
% A munka  előzményei és kiindulási állapota
% \newpage
\subsection{Bevezető}
\label{sec:bevezeto}

Hálózat kiépítés és karbantartás költséges => nem éri meg a kihasználatlan kapacitás

Telített hálózaton lassú adattovábbítás => szigorú 5G Qos előírások nem teljesülnek

A hálózatot modellezve felmérhető annak teljesítménye => win-win a usernek és üzemeltetőnek

Ez optimalizálási feladat, amit jó lenne lineáris időben végezni

\subsection{Elméleti összefoglaló}

Egy hálózatot reprezentálhatunk egy irányított gráffal. Legyen ez a gráf

$$G = \lbrace V, E, w \rbrace \text{ ahol } V \text{ csúcshalmaz a hálózati eszközök halmaza },$$
$$E = \lbrace e = (v_i, v_j) \, \vert \, v_i, v_j \in V \rbrace \text{ élhalmaz az eszközök közötti összeköttetések halmaza },$$
$$\forall e \in E \quad w(e) \text{ az $e$ él súlya, az adat átküldési idejével arányos }.$$

A legrövidebb út problémánál adott a forrás és a cél eszköz, amelyek között útvonalat keresünk.
Jelölje ezeket $s$ és $t$ csúcs.
Egy $s$ és $t$ közöti körmentes utat megadhatunk \eqref{eq:1.1} egyenlettel.

\begin{equation}
  \forall e \in E \quad x(e) = \begin{cases}
    1, & \text{ha az $e$ él része az útvonalnak}\\
    0, & \text{különben}
  \end{cases} \label{eq:1.1}
\end{equation}

A továbbiakban jelölje $\outedge{v}$ a $v$ csúcsból kifutó élek halmazát,
$\inedge{v}$ pedig a $v$ csúcsba befutó élek halmazát.
Az $x(e)$ függvény akkor ad meg érvényes útvonalat, ha teljesül \eqref{eq:1.2a}, a folyammegmaradás törvénye,
tehát hogy $s$ csúcsból az útnak egy éle fut ki, $t$ csúcsba egy éle fut be,
a többi csúcsban pedig a ki- és befutó élek száma megegyezik.
A legrövidebb út összsúlya így az \eqref{eq:1.2b} kifejezés.

\begin{subequations}
  \begin{equation}
    \forall v \in V \quad \sum_{e}^{\inedge{v}}x(e) - \sum_{e}^{\outedge{v}}x(e) = \begin{cases}
      1, & v = s \\
      -1, & v = t\\
      0, & \text{különben}
    \end{cases} \label{eq:1.2a}
  \end{equation}
  \begin{equation}
    \min_{x(e)} \sum_{e}^{E} x(e) \, W(e) \label{eq:1.2b}
  \end{equation}
\end{subequations}

A legrövidebb út probléma megoldására használható az IBM ILOG (CPLEX) Optimzer CPLEX,
ami egy a Simplex módszerre épülő szoftvercsomag.
Segítségével lineáris programozási optimalizálási \linebreak problémákat modellezhetünk,
az elkészített modellre a szoftverrel optimális stratégia kereshető.
A modellt döntési változók (röviden változók), kényszerek és egy célfüggvénny alkotja.
A döntési változók egészek, valós számok, vagy logikai 1 és 0 típusúak lehetnek.
Az egyes változók értékkészletére tehetünk további megkötést.
A kényszerek lineáris egyenletek vagy egyenlőtlenségek, amikben az ismeretlenek a döntési változók.
Egy kényszer a válozók által felvehető értékeket korlátozza.
A probléma megoldása egy olyan változó érték hozzárendelés, amelyre az összes kényszer teljesül.
A célfüggvény a változók egy lineáris kombinációja.
Az optimális megoldás az, amely esetén a célfüggvényt értéke - a problémától függően - maximális vagy minimális.

Bár a szoftvert komplexebb problémákra tervezték, a legrövidebb út probléma jó
alapot szolgáltat a bonyolultabb hálózatokkal kapcsolatos problémák megoldásához.
Az $x(e)$ függvény értékeit keresve azokat döntési változóként kell fölvenni a modellbe.
A típusuk logikai 1 vagy 0.
Az \eqref{eq:1.2a} egyenlet egy kényszer, a célfüggvény az \eqref{eq:1.2b} kifejezés.

A legrövidebb útvonal ismerete azonban még nem feltétlenül elégséges,
figyelembe kell venni, hogy az egyes kapcsolatok megszakadhatnak.
Ha az útvonal mentén egy kapcsolat kiesik, e-miatt megszakadhat a kommunikáció.
Megoldást jelenthet a legrövidebb út mellett egy védelmi útvonal keresése,
amely \linebreak éldiszjunkt a fő útvonallal.
Ha a fő útvonalon megszakad a kapcsolat, a védelmi úton folytatódhat a kommunikáció.

Az 5G hozzáférési hálózatok által támogatott, hogy egy mobil eszköz (UE, mint User Equipment)
egyszerre több gNodeB-vel\footnote{
  Az 5G hozzáférési hálózat eleme, rádióadóval felszerelt állomás, ami a mobil eszközökkel közvetlen kommunikál,
  adatkapcsolatot biztosít az eszköz és a hálózat többi eleme között.
  }
tartson fenn aktív kapcsolatot, így növelve a szolgáltatás minőségét és az elérhető sávszélességet. \cite{multiconnectivity}
Megjelentek olyan IP alapú protokollok, például az MC-TCP \cite{mptcp} amik \linebreak támogatják két eszköz között több kapcsolat felállítását, különböző interfészeken.
A többutas kapcsolatot elfedik, az alkalmazás rétegből a többszörös csatlakozás nem látszik.
Egyszeres kapcsolódás esetén, ha egy UE forgalma $D$, akkor e-mellé egy ugyanekkora éldiszjunkt védelmi útvonalat kell nyilvántartani.
Azonban ha egy UE $n$ darab kapcsolaton osztja el a forgalmát, $n - 1$ kapcsolaton küldhet egyenként $\frac{D}{n - 1}$ forgalmat, a
fennmaradó kapcsolatot védelemként használva azon csak $\frac{D}{n - 1}$ forgalomra lesz szükség, ha egy kapcsolat megszűnését feltételezzük.

%==================================================================
\section{Az elvégzett munka és az eredmények ismertetése}
\label{sec:az-elvegzett-munka}

\subsection{Optimális útvonalak többfelhasználós esetben}
\label{sec:multiuser}

Több felhasználó esetén a felhasználókhoz külön külön útvonalat kell megállapítani.
Legyen a felhasználók halmaza $U$.
Mivel a felhasználók helyzete más, és különböző végpontokkal kommunikálnak, ezért külön forrás és cél csomópont, 
és külön $x(e)$ függvény tartozik hozzájuk, jelölje ezeket $u \in U \quad s_u, t_u, x_u(e)$,
továbbá egyes felhasználóknak eltérő mennyiségű folyam igényei lehetnek, ezt jelölje $D_u$.

Figyelembe kell venni, hogy az egyes elemek közötti kapcsolatok véges sávszélességet tudnak csak biztosítani.
Jelölje $e \in E \quad c(e)$ függvény egy összeköttetés maximális kapacitását.
A \eqref{eq:2.1b} kényszer biztosítja, hogy egy élen ne keletkezzen nagyobb forgalom, mint amennyit az el tud szállítani.

A folyammegmaradási kényszer nem változik, de minden felhasználóra és minden csúcsra definiálni kell ezt mutatja a \eqref{eq:2.1a}.
A \eqref{eq:2.1c} kifejezéssel leírt célfüggvénnyel optimálisnak tekinthető megoldás kapható:
az összes felhasználó utvonalának összsúlyának összege minimális.

\begin{subequations}
  \begin{equation}
    \forall u \in U \quad \forall v \in V \quad \sum_{e}^{\inedge{v}}x_u(e) - \sum_{e}^{\outedge{v}}x_u(e) = \begin{cases}
      1, & v = s_u \\
      -1, & v = t_u\\
      0, & \text{különben}
    \end{cases} \label{eq:2.1a}
  \end{equation}
  \begin{equation}
    \forall e \in E \quad \sum_{u}^{U} D_u \cdot x_u(e) \leq c(e) \label{eq:2.1b}
  \end{equation}
  \begin{equation}
    \min_{x_u(e)} \sum_{u}^{U} \sum_{e}^{E} x_u(e) \, w(e) \label{eq:2.1c}
  \end{equation}
\end{subequations}

\subsection{Éldiszjunkt védelmi útvonalak}
\label{sec:edgedisjunct}

A védelmi útvonalak repreyentálhatók az $x_u(e)$ függvénnyekkel azonos módon.
A védelmi és fő útvonalak megkülönböztetése érdekében a védelmi utakat jelöljék az $y_u(e)$ függvények.
A \eqref{eq:2.1a} egyenlet megfelelő védelmi utak mellett is.
A mintájára felírható \eqref{eq:2.2a} kényszer,
ami a védelmi útvonalakra mondja ki a folyammegmaradási kényszert.
A \eqref{eq:2.2b} kényszer biztosítja, hogy akár fő, akár védelmi útvonal eleme egy él,
a rajta átmenő folyam ne haladja meg a kapcsolat kapacitását.

\begin{subequations}
  \begin{equation}
    \forall u \in U \quad \forall v \in V \quad \sum_{e}^{\inedge{v}}y_u(e) - \sum_{e}^{\outedge{v}}y_u(e) = \begin{cases}
      1, & v = s_u \\
      -1, & v = t_u\\
      0, & \text{különben}
    \end{cases} \label{eq:2.2a}
  \end{equation}
  \begin{equation}
    \forall e \in E \quad \sum_{u}^{U} D_u \left( x_u(e) + y_u(e) \right) \leq c(e) \label{eq:2.2b}
  \end{equation}
\end{subequations}

\newpage

Ahhoz, hogy mindegyik $u \in U$ felhasználóra a hozzátartozó fő és védelmi útvonalak éldiszjunktak legyenek az kell,
hogy minden élre $x_u(e)$ és $y_u(e)$ értékek közül csak egyik lehessen egy,
hisz ekkor az él csak az egyik útvonalnak lehet a tagja. Ezt fejezi ki a \eqref{eq:2.3}-as egyenlőtlenség.

\begin{equation}
  \forall u \in U \quad \forall v \in V \quad x_u(e) + y_u(e) \leq 1 \label{eq:2.3}
\end{equation}

A célfüggvénynek a fő és védelmi útvonalak együttes összsúlyát kell minimalizálnia,
azonban a védelmi út kevésbé fontos, mint a fő útvonal, hiszen előbbi csak meghibásodás esetén
használandó, míg utóbbin folyamatos az adatforgalom.
Például ha egy felhasználó számára két éldiszjunkt út áll rendelkezésre, egyik összsúlya $w_1$, másiké $w_2$,
és $w_1 < w_2$, akkor az első utat lenne érdemes főútvonalnak választani, a másodikat pedig védelemnek.
Ezért a célfüggvényben a védelemi út összsúlya egy $\alpha \in (0, 1]$ szorzóval számítson bele az összegbe.
Így a két lehetséges eset a célfüggvény értékére \eqref{2.4a} és \eqref{2.4b}.
Az előbbiből kivonva az utóbbit áll elő \eqref{2.4c}.
A \eqref{2.4d} egyenletből, $w_1, w_2$ és $\alpha$ értékei miatt látható,
hogy a CPLEX helyesen fogja megválasztani a fő és cél útvonalt.

\begin{subequations}
  \begin{equation}
    W' = w_1 + \alpha \, w_2 \label{2.4a}
  \end{equation}
  \begin{equation}
    W = w_2 + \alpha \, w_1 \label{2.4b}
  \end{equation}
  \begin{align}
    W - W' &= w_2 - \alpha \, w_2 + \alpha \, w_1 - w_1 \substack{\text{saf}} \label{2.4c} \\
    W - W' &= \left( 1 - \alpha \right)\left( w_2 - w_1 \right) > 0 \Rightarrow W' < W \label{2.4d}
  \end{align}
\end{subequations}

A célfüggvény általános alakja a \eqref{eq:2.5} kifejezés.

\begin{equation}
  \min_{x_u(e), \, y_u(e)} \left \lbrace \sum_{u}^{U} \sum_{e}^{E} \left( x_u(e) + \alpha \, y_u(e) \right) w(e) \right \rbrace \label{eq:2.5}
\end{equation}

TODO: FOLYAM ABRAK

\subsection{Egyszerű 5G hálózat modellezése}
\label{sec:5gnetwork}

\usetikzlibrary{arrows.meta}
\begin{figure}[h]
  \centering
  \begin{tikzpicture}
    % Nodes
    \node[circle,minimum width=0.5cm,draw,label={UE$_1$}] (ue1) at (0,4) {};
    \node[circle,minimum width=0.5cm,draw,label={UE$_2$}] (ue2) at (0,2) {};
    \node[circle,minimum width=0.5cm,draw,label={UE$_3$}] (ue3) at (0,0) {};

    \node[circle,minimum width=0.5cm,draw,label={gNodeB$_1$}] (gb1) at (3,3.5) {};
    \node[circle,minimum width=0.5cm,draw,label=below:{gNodeB$_2$}] (gb2) at (3,.5) {};

    \node[circle,minimum width=0.5cm,draw,label=right:{gateway}] (gw) at (6,2) {};

    \draw[-{Stealth[scale=1.5pt]}] (ue1) -- (gb1);
    \draw[-{Stealth[scale=1.5pt]}] (ue1) -- (gb2);

    \draw[-{Stealth[scale=1.5pt]}] (ue2) -- (gb1);
    \draw[-{Stealth[scale=1.5pt]}] (ue2) -- (gb2);

    \draw[-{Stealth[scale=1.5pt]}] (ue3) -- (gb1);
    \draw[-{Stealth[scale=1.5pt]}] (ue3) -- (gb2);

    \draw[-{Stealth[scale=1.5pt]}] (gb1) -- (gw);
    \draw[-{Stealth[scale=1.5pt]}] (gb2) -- (gw);
  
    % Edges
  \end{tikzpicture}
  \caption{Az egysyerűsített 5G hálózat topológiája.} \label{fig:1}
\end{figure}


\newpage

\subsection{Összefoglalás}
\label{sec:osszefoglalas}

\newpage
 
%==================================================================
\section{Irodalom, és csatlakozó dokumentumok jegyzéke}
\label{sec:irod-es-csatl}

\begin{thebibliography}{9}
\label{sec:tanulm-irod-jegyz}

\bibitem{multiconnectivity} S. A. Busari, R. Mumtaz and J. Gonzalez,
\emph{MULTI-CONNECTIVITY IN 5G NEW RADIO STANDARDS},
2020. november 30.
Elérhető: \url{https://www.standardsuniversity.org/e-magazine/december-2020/multi-connectivity-in-5g-new-radio-standards/}
2023. november 24-én

\bibitem{mptcp} Olivier Bonaventure, Mark Handley, Costin Raiciu,
\emph{An Overview of Multipath TCP}
2012. október
Elérhető: \url{https://www.usenix.org/system/files/login/articles/login1210_bonaventure.pdf}
2023. november 24-én

\end{thebibliography}

Itt jegyezném meg, hogy a tanulmányozott irodalmat hivatkozni kell a
szövegben.  Szükség esetén többször is.  Az irodalomjegyzék célja
(lásd \aref{sec:tanulm-irod-jegyz} fejezetet) ugyanis
kettős\footnote{Akárcsak ennek a fejezet hivatkozásnak, ami a
  \texttt{$\backslash$aref babel} parancsot demonstrálja}:
\begin{enumerate}
\item Az olvasó tájékoztatása, hogy a dokumentumban ki nem fejtett
  dolgoknak, a tudottnak vélt ismereteknek hol lehet bővebben
  utánanézni, így ott kell meghivatkozni az irodalmat~\cite{eco,
    esterhazy}, ahová az irodalom kapcsolódik.
\item Megmutatni a tárgyfelelosnek/konzulesnek az elolvasott irodalom
  mennyiségét
\end{enumerate}

Javasoljuk, hogy a hallgatók tanulmányozzák, hogyan néznek ki a
hivatkozások a villamosmérnöki/informatikai szakma vezető szakmai
folyóirataiban megjelenő cikkekben.  Ebben a témavezető is biztosan
tud segíteni.  A hivatkozás teljességére és egyértelműségére tessék
ügyelni.  Például, ha egy könyvnek több, eltérő kiadása is van, akkor
azt is meg kell jelölni, hogy melyik kiadásra hivatkozunk.  A webes
hivatkozások problémásak szoktak lenni, de manapság egyre több az
olyan dokumentum, ami csak weben lelhető fel, ezért használatuk nem
zárható ki. Itt is törekedni kell azonban a pontosságra és a
visszakereshetőségre. A weben található dokumentumoknak is van címe,
szerzője, illetve érdemes megadni a letöltés/olvasás időpontját is,
hiszen ezek a dokumentumok idővel megváltozhatnak.

A wikipédiás hivatkozások használata nem javasolt, mert a wikipedia
másodlagos forrás.  Tájékozodjuk a wikipédián, de aztán olvassuk el az
adott oldalhoz megadott hivatkozásokat is.  A wikipedián külön szócikk
foglalkozik azzal, hogy miért nem szerencsés tudományos munkákban a
wikipédiára hivatkozni \cite{wikipedia}.

Nem publikus dokumentumok hivatkozása nem javasolt és csak kivételes
helyzetben elfogadható!

%==================================================================
\subsection{A csatlakozó dokumentumok jegyzéke}
\label{sec:csat-irod}

<A munka ezen beszámolóba be nem fért eredményeinek (például a forrás
fájlok, mindenképpen csatolni akart forráskód részlet, felhasználói
leírások, programozói leírások (API), stb.) megnevezése,
fellelhetőségi helyének pontos definíciója, mely alapján a az
erőforrás előkereshető -- értelemszerűen nem nyilvános dokumentumok
hivatkozása nem elfogadható.>

\end{document} 

%%% Local Variables: 
%%% mode: latex 
%%% TeX-master: t 
%%% End:

